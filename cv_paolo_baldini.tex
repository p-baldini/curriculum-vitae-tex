\documentclass{mandrabcvstyle}

\usepackage{multicol}

\begin{document}

\begin{candidate}
    \photo{res/photo.jpg}
    \name{Paolo Baldini}
    \birthday{1997-09-20}
    \location{Mercato Saraceno (FC), Emilia Romagna, Italy}
    \email{paolo.baldini.phd@gmail.com}
    \phone{(+39) 340 1622863}
    \drivingLicence{B}
    \home{Website}{https://p-baldini.github.io}
    \github{GitHub}{https://github.com/p-baldini}
    \scholar{Scholar}{https://scholar.google.com/citations?user=vUmVyzMAAAAJ&hl=it&oi=ao}
    \university{University of Bologna}{https://www.unibo.it/sitoweb/p.baldini}
    \linkedin{Linkedin}{https://www.linkedin.com/in/baldini-paolo}
\end{candidate}

\section{Education}

\experience
    {Doctor of Philosophy (PhD) in Computer Science and Engineering}
    {A.Y. 22/23 -- 24/25}
    {University of Bologna}
    {
        Studies on continual adaptation of robotic systems
    }

\experience
    {Master of Computer Science and Engineering}
    {A.Y. 19/20 -- 20/21}
    {University of Bologna}
    {
        $110/110$ with honours\\
        Thesis: \href{https://amslaurea.unibo.it/25396}{\itshape Online adaptation of robots controlled by nanowire networks}
    }

\experience
    {European Master on Advanced Robotics -- Erasmus+}
    {10/2020 -- 02/2021}
    {Warsaw University of Technology}{4.84/5.00}

\experience
    {Bachelor of Computer Science and Engineering}
    {A.Y. 16/17 -- 18/19}
    {University of Bologna}
    {
        $104/110$\\
        Thesis: \href{https://amslaurea.unibo.it/19108}{\itshape Progettazione e sviluppo di un modulo sensoristico e di comunicazione per la piattaforma robotica AlpaBot}
    }

\experience
    {High School Diploma in Electronics and Electrotechnics}
    {A.Y. 11/12 -- 15/16}
    {Istituto Tecnico Industriale Statale "Blaise Pascal"}
    {$70/100$\\ Project: {\itshape Launchpad: electrical and software design}}

\section{Publications}

\experience{Compensating design-time uncertainty through combined Offline Design and\\\phantom{a} Online Adaptation of robot control software}{submitted}{}{}

\vspace{-\baselineskip}
\experience{Online Adaptation of heterogeneous robot swarms to a dynamic task}{submitted}{}{}

\vspace{-\baselineskip}
\experience{Unraveling creativity through variability: a comparison of LLMs and humans in\\\phantom{a} an educational Q\&A scenario}{2026}{}{}

\vspace{-\baselineskip}
\experience{On the LLM robustness in a simulated conversational XR scenario: a preliminary semantic analysis}{2026}{}{}

\vspace{-\baselineskip}
\experience{Fault Recovery through Online Adaptation of Boolean Network robots}{2025}{}{}

\vspace{-\baselineskip}
\experience{Cooperation in evolved modular soft robots}{2025}{}{}

\vspace{-\baselineskip}
\experience{Impact of LLM--assisted coding in creativity and robustness of robot controllers}{2025}{}{}

\vspace{-\baselineskip}
\experience{Cell--Cell interactions: how coupled Boolean Networks tend to criticality}{2025}{}{}

\vspace{-\baselineskip}
\experience{Emergent dynamics in heterogeneous life-like Cellular Automata}{2025}{}{}

\vspace{-\baselineskip}
\experience{An investigation of graceful degradation in Boolean network robots\\\phantom{a} subject to online adaptation}{2024}{}{}

\vspace{-\baselineskip}
\experience{On the performance of online adaptation of robots controlled by nanowire networks}{2023}{}{}

\vspace{-\baselineskip}
\experience{Online adaptation of robots controlled by nanowire networks: A preliminary study}{2023}{}{}

\vspace{-\baselineskip}
\experience{Reservoir Computing in robotics: a review}{2022}{}{}

\section{Academic experience}

\experience
    {Program committee of international conferences}
    {}
    {
        Part of the Program Committee of the ``First International Workshop on\\\phantom{a} Human-AI Collaborative Systems (HAIC)'' \hfill 2025 \newline
        Part of the Program Committee of ``The International Conference for Artificial Life (ALife)'' conference \hfill 2025}
    {}

\experience
    {Reviewer for international conferences}
    {}
    {
        Reviewer for the ``The International Conference for Artificial Life (ALife)'' \hfill 2024 -- 2025 \newline
        Reviewer for the ``International Workshop on Artificial Life and Evolutionary Computation (Wivace)'' \hfill 2024}
    {}

\experience
    {Thesis co-supervisor}
    {}
    {
        Eleonora Bertoni, ``Innate reflexes for safe Online Adaptation of robots'' \hfill ongoing \newline
        Benedetta Pacilli, ``Mechanisms for Online Adaptation in robots: a comparative study'' \hfill 2025 \newline
        Noemi Valentini, ``Evolution in materia: esperimenti con le nanowire networks'' \hfill 2024 \newline
        Filippo Benvenuti, ``Self-Assembly in Voxel-Based Robots'' \hfill 2024}
    {}

\section{Some work experiences}

\experience
    {Academic Tutor}
    {12/2025 -- present}
    {\href{https://www.unibo.it}{University of Bologna}}
    {
        Tutor at several high-schools for the university orientation project ``\textit{Nei panni di un robot: come programmare il comportamento di un robot}''.
    }

\experience
    {Course Tutor}
    {03/2023 -- present}
    {\href{https://www.unibo.it}{University of Bologna}}
    {
        Tutor at the courses of: \textit{Computer Architectures} @ Bachelor of Computer Science and Engineering; \textit{Algorithms and Data Structures} @ Bachelor of Information Science for Management.
    }

\experience
    {Doctoral Researcher}
    {10/2022 -- present}
    {\href{https://www.unibo.it}{University of Bologna}}
    {
        Studies on continual adaptation of robotic systems.
    }

\experience
    {IT consulting}
    {06/2022 -- 09/2022}
    {\href{https://www.gaiagsat.eu}{GaiaG S.R.L.}}
    {
        Debug and ad-hoc specialized tasks in the area of decision support systems.
    }

\experience
    {C++ programmer}
    {10/2018 -- 12/2018}
    {\href{https://elements-ic.com}{Elements S.R.L.}}
    {
        Development of software systems for the analysis of data collected from a current amplifier for nano-pores and single channel electrophysiology applications.
    }

% \experience
%     {Stadium steward}
%     {11/2015 -- 05/2016}
%     {\href{https://www.vivaticket.com}{Vivaticket (ex BestUnion)}}
%     {Different responsibilities, such as: crowd management, security control and welcoming.}

\experience
    {Electrical expert assistant}
    {06/2015 -- 08/2015}
    {Studio EQUA}
    {Electric and architectural plants design. Archive management.}

% \experience
%     {Summer entertainer}
%     {06/2013 -- 09/2015}
%     {Centro estivo molto sportivo!}
%     {Organization assistant. Entertainment and welcoming.}

\section{Languages}

\begin{multicols}{3}
    \languageknowledge{Italian}{6}{Native}
    \languageknowledge{English}{5}{C1}
    \languageknowledge{Spanish}{2}{A2}
    \languageknowledge{French}{2}{A2}
    \languageknowledge{Turkish}{1}{A1}
\end{multicols}

\section{Projects (a few)}

% \experience
    % {Online adaptation of robots}
    % {2022}{}
    % {
    %     Continuous adaptation of a robotic system to some tasks and environments:\\\hspace*{9pt}
    %     (1) collision avoidance, (2) area avoidance, (3) T-maze.
    % }

\experience
    {Memristive device simulator}
    {2021}{}
    {
        Optimization and packaging of the simulator of a memristive, nanometric device for a physic research group in Turin.
    }

% \experience
%     {Distributed warehouse services}
%     {2021}{}
%     {Distributed heterogeneous systems using Jade framework and Jason language.}

% \experience
%     {Rules induction algorithm}
%     {2020}{}
%     {CN2 learning algorithm for rules induction.}

\experience
    {Trajectory estimator}
    {2020}{}
    {
        Matlab script for the estimation of a ball trajectory from captured images (i.e., Computer Vision).
    }

\experience
    {Trajectory reconstructor}
    {2020}{}
    {
        Matlab script for the reconstruction of a camera trajectory from captured images (i.e., Computer Vision).
    }

% \experience
%     {Web Kanban}
%     {2020}{}
%     {Kanban website for collaborative, agile project management.}

% \experience
%     {Particle simulator}
%     {2020}{}
%     {Simulator of the physic motion of particles.}

\experience
    {Compiler for a custom programming language}
    {2020}{}
    {
        Design of a compiler for a Functional, Object-Oriented programming language,\\\hspace*{9pt}
        for and with the collaboration of a University Professor.
    }

% \experience
%     {IoT system: hardware and software creation}
%     {2019}{}
%     {
%         Collaborated in the hardware creation of IoT devices,
%         like connection of sensors and power calibration,\\\hspace*{9pt}
%         for and with the collaboration of a University Professor.
%         Created the control software for their interaction.
%     }

% \experience
%     {CC-Tan game}
%     {2018}{}
%     {Java version of the Android video game CC-Tan.}

\section{Competencies}

\begin{multicols}{2}
\subsection{General}
\skill{Multi-paradigm programming}{6}{}
\skill{Embedded systems \& robotics}{6}{}
\skill{Electrical design and construction}{5}{}
\skill{Distributed systems \& IoT}{3}{}
\skill{Full-stack web development}{2}{}
\skill{Others\dots}{-1}{}
\vfill
\subsection{Tools}
\skill{Git}{6}{}
\skill{Build automation (Gradle, CMake, etc.)}{5}{}
\skill{Documentation (UML, OpenAPI, *doc)}{5}{}
\skill{Docker}{5}{}
\skill{Office suites (Word, Excel, PowerPoint)}{5}{}
\skill{Others\dots}{-1}{}
\end{multicols}
\vspace{-2\baselineskip}

\section{Soft skills}

\begin{multicols}{2}

\skill
    {Creativity and problem solving}{-1}
    {
        Loving logic and competitive games since my since childhood.
        Enhanced during university.
    }

\skill
    {Organization}{-1}
    {
        My engineering background required a high level of data and task organization.
    }

\skill
    {Self-improvement}{-1}
    {
        I constantly try to learn more and use information to produce something new.
        My Ph.D. is a clear example of this attitude.
    }

\skill
    {Group work and leadership}{-1}
    {
        My academic employment required constant collaboration and supervision.
        Clear outcomes of this aspect are the various publications and theses that I (co-)authored and / or supervised. 
    }

\skill
    {Autonomy and initiative}{-1}
    {Working with many different groups sometimes requires taking the initiative and operating autonomously.}

\end{multicols}

\begin{declaration}
    In compliance with the Italian Legislative Decree no. 196 dated 30/06/2003,
    and in agreement with the GDPR (EU Regulation 2016/679),
    I hereby authorize the recipient of this document to use
    and process my personal details for the purpose of recruiting
    and selecting staff and I confirm to be informed of my rights
    in accordance with art. 7 of the above-mentioned decree.
    \signature{res/signature.png}
\end{declaration}

\end{document}
